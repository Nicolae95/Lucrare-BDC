\section*{Introduction to Stored procedures}
\phantomsection

Stored procedures have been viewed as the de facto standard for applications to access and manipulate database information through the use of codified methods, or “procedures.” This is largely due to what they offer developers: the opportunity to couple the set-based power of SQL with the iterative and conditional processing control of code development. Developers couldn’t be happier about this; finally, instead of writing inline SQL and then attempting to manipulate the data from within the code, developers could take advantage of:\\
\textbf{Familiar Coding Principles}
\begin{enumerate}
\item Iterative Loops
\item Conditionals
\item Method Calls (the stored procedure itself is built and similarly called like a method)
\end{enumerate}
\textbf{One-time, One-place Processing}
\begin{enumerate}
\item Instead of having inline SQL code spread throughout the application, now sections of SQL code can be encapsulated into chunks of named methods that are easily identifiable and accessible all within one location – the “Stored Procedure” folder of the database.
\item All complex data processing can now be performed on the server, allowing the client processing to focus more on presentation rather than manipulation of data.
\end{enumerate}

Of course, just because something is popular doesn’t always mean that it’s the best tool in all situations. The efficiency, efficacy and utility of Stored Procedures, just like the implementation of all programming languages and platforms, are all dependent on the needs of the client and the subsequent architecture of the application.



\clearpage